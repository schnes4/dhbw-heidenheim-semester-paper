% !TeX root = ../main.tex

\addchap{\appendixPhrase}

\section{Arduino-Programmcode}
\begin{lstlisting}[language=C, label={lst:arduino_selfmade_interrupts}, caption=Programm-Code zur Erstellung des Self-Made Interrupts]
#define ITR_PIN 7
#define IR_SENS 9

unsigned long time;
int falling = true;

void setup() {
  Serial.begin(38400);

  pinMode(IR_SENS, INPUT);
  pinMode(ITR_PIN, OUTPUT);

  digitalWrite(ITR_PIN, LOW);
}

void loop() {
  int sensor = digitalRead(IR_SENS);
  
  if (sensor == LOW) {
    if (falling) {
      digitalWrite(ITR_PIN, HIGH);
      delay(5);
      digitalWrite(ITR_PIN, LOW);

      Serial.println(5);

      falling = false;
    }
  }
  else {
    falling = true;
  }

  Serial.println(sensor);
}
\end{lstlisting}

\begin{lstlisting}[language=C, label={lst:modified_twi_h}, caption=Modifizierte \emph{twi.h} um \ac{I2C} im \emph{Fast-Mode} zu betreiben]
/*
  twi.h - TWI/I2C library for Wiring & Arduino
  Copyright (c) 2006 Nicholas Zambetti.  All right reserved.

  This library is free software; you can redistribute it and/or
  modify it under the terms of the GNU Lesser General Public
  License as published by the Free Software Foundation; either
  version 2.1 of the License, or (at your option) any later version.

  This library is distributed in the hope that it will be useful,
  but WITHOUT ANY WARRANTY; without even the implied warranty of
  MERCHANTABILITY or FITNESS FOR A PARTICULAR PURPOSE.  See the GNU
  Lesser General Public License for more details.

  You should have received a copy of the GNU Lesser General Public
  License along with this library; if not, write to the Free Software
  Foundation, Inc., 51 Franklin St, Fifth Floor, Boston, MA  02110-1301  USA
*/

#ifndef twi_h
#define twi_h

  #include <inttypes.h>

  //#define ATMEGA8

  #ifndef TWI_FREQ
  #define TWI_FREQ 400000L
  #endif

  #ifndef TWI_BUFFER_LENGTH
  #define TWI_BUFFER_LENGTH 32
  #endif

  #define TWI_READY 0
  #define TWI_MRX   1
  #define TWI_MTX   2
  #define TWI_SRX   3
  #define TWI_STX   4
  
  void twi_init(void);
  void twi_disable(void);
  void twi_setAddress(uint8_t);
  void twi_setFrequency(uint32_t);
  uint8_t twi_readFrom(uint8_t, uint8_t*, uint8_t, uint8_t);
  uint8_t twi_writeTo(uint8_t, uint8_t*, uint8_t, uint8_t, uint8_t);
  uint8_t twi_transmit(const uint8_t*, uint8_t);
  void twi_attachSlaveRxEvent( void (*)(uint8_t*, int) );
  void twi_attachSlaveTxEvent( void (*)(void) );
  void twi_reply(uint8_t);
  void twi_stop(void);
  void twi_releaseBus(void);

#endif
\end{lstlisting}
%!TEX root = ../../main.tex

\chapter{Einleitung}
Im Rahmen dieser Studienarbeit soll die Messung und Analyse von Unwuchten, welche auf einen Flugzeugmotor wirken, durchgeführt werden. Unwuchten bei Flugzeugen werden in den meisten Fällen von den Propellern erzeugt, wodurch zum Teil erhebliche Vibrationen entstehen, was zum vorzeitigen Verschleiß von Triebwerken führt, sowie auch den Flugkomfort unangenehm beeinflusst. Um diese Unwuchten zu bestimmen, wird ein Mikrocontroller-Board der Marke \textit{Arduino}, sowie verschiedene Sensoren verwendet. In einem ersten exemplarischen Modell, wird das Konzept der Unwuchtmessung an einem elektrischen Motor evaluiert, welcher hauptsächlich im Bereich des Modellfluges Anwendung findet. Bei einem positiven Ausgang des Experimentes, wird die Vorgehensweise auf einen realen Flugzeugmotor überführt, eventuelle Anpassungen vorgenommen und dementsprechend getestet. Das Ergebnis der Messungen, so wie der anschließenden Analyse, wird letztendlich auf einem TFT-Display des Mikrocontroller-Boards dargestellt.
%!TEX root = ../../main.tex

\chapter{Einleitung}
Im Rahmen dieser Studienarbeit soll die Messung und Analyse von Unwuchten, welche auf einen Flugzeugmotor wirken, durchgeführt werden.
Unwuchten bei Flugzeugen werden in den meisten Fällen von den Propellern erzeugt.
Durch diese Unwuchten können zum Teil erhebliche Vibrationen entstehen, welche neben einem vorzeitigen Verschleiß von Triebwerk und Flugzeugrumpf, auch unangenehme Auswirkungen auf den Flugkomfort haben.
Um diese Unwuchten detektieren und lokalisieren zu können, werden zwei Mikrocontroller-Boards der Marke \textit{Arduino}, sowie verschiedene Sensoren verwendet.

In einem ersten exemplarischen Modell, wird das Konzept der Unwuchtmessung an einem elektrischen Motor evaluiert, welcher hauptsächlich im Bereich des Modellfluges Anwendung findet.
Bei einem positiven Ausgang des Experimentes, wird die Vorgehensweise auf einen realen Flugzeugmotor überführt, eventuelle Anpassungen vorgenommen und entsprechend getestet. 
Das Ergebnis der Messungen, so wie der Analyse, soll letztendlich auf einem TFT-Display des Mikrocontroller-Boards dargestellt werden.